\section{Prototype infrastructure}\label{sec:background}

%/* speak here about messaging queuing, loosly coupled architecture, artifact
% based approach. */
The system has been designed such that the whole architecture is loosely
coupled. Messages are used for maintaining status information
(running/stopping) of a workflow instance. The system mainly consists of three
main components - message, task and a repository. A
detailed description of each component is given below -

\begin{itemize}
  \item{\emph{Message}: A message contains a run id and the
  status information of a workflow instance. It allows the workers to obtain the current state of
  the instance and trigger the next job, provided the instance is
  still running. The message contains no information about the state of the
  instance.}
  \item{\emph{Task}: A workflow is a set of tasks, either in sequence or in
  parallel. Each task represents a granular activity that is atomic, i.e. a
  task can modify a single key(entity). Each task has a corresponding -Negation
  Task-. A negation task is the one that nullifies the effects of the original
  task in case of workflow failure, to bring the system state back to
  its original state. Each task that modifies a key-value pair, must have a
  negation task. A simple negation task can be one that restores the previous
  version of a key-value pair. Although, as we discussed
  earlier that real world tasks may not be reversible, we let the users of this
  system design custom negation tasks to provide flexibility.}
  \item{\emph{Repository}: The repository stores two types of data - the actual
  data that is modified by business logic; and the metadata associated with
  workflows - workflow definition, task definition, run state, etc. To ease the
  complexity of the design, we can also have two separate repositories. The
  repository is accessed by workers to obtain the run state of a workflow
  instance or the data that can be modified by the instance.}  
\end{itemize}