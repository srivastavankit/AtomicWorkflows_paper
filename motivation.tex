\section{Motivation} \label{sec:motivation}

%/* Write about the problem of overhead in distributed transactions. Actual
% workflows are isolated from each other's execution. */



%/* 

%In our project we tackle one of the main challenges in WSNs, namely energy
% efficiency. Sensors have a limited amount of power. In the absence of any
% tasks, a node wastes its energy waiting for transmissions. Making a node
% sleep (i.e., consume less power) in inactive periods is a well-studied way to
% mitigate energy waste \cite{1}. However, we consider the effect of the sleep
% cycle duration on energy consumption. We show a motivating example in
% Figure~\ref{fig:motivating}. In it we consider two scenarios where no data
% exist to be received. The first scenario has a sleep cycle duration equal to
% half of that of the second scenario. The black rectangles denote the periods
% of polling for buffered packets. It is apparent that scenario one consumes
% more energy since it is switching to poll for received packets more often.
% The effect of sleep cycle duration extends to scenarios with active
% receptions and can be intuitively derived.  Figure~\ref{fig:longevity} shows
% the effect of sleep cycle times on the longevity of nodes. It is shown that a
% sleep time of 1.0 seconds increase longevity by a factor of 10000\% compared
% to 0.01 seconds.

%Given our motivation, the main objective of our work is to maximize sleeping
% times. However, there is a trade-off between sleep times and QoS requirements
% (we consider delay); a longer sleep cycle duration translates to a larger
% delay times. A query-based framework is considered for our implementation
% \cite{2}. We will use ZigBee, based on IEEE 802.15.4 \cite{3}, as our
% system's infrastructure. The objective of our work is threefold: First, we
% will deploy a query-based WSN using Arduino micro-controllers \cite{17} and
% XBEE modules \cite{18}. Second, we will implement the query-based scheme over
% QualNet simulator~\cite{16}. Using the deployment and simulation framework we
% can study the system for the effects of sleep cycle duration on QoS and
% energy consumption. Third, a mathematical model will be constructed to
% capture QoS and energy characteristics. From the model, a formula for
% determining maximum sleeping times is derived. The resulted formula will be
% tested in the deployment (as a proof-of-concept prototype) and in simulation (to investigate scalability).

%The main contribution of our work is to produce a mathematical model that can
% be used to capture QoS with respect to sleep behavior in WSNs. This model can
% then be used to derive suitable sleeping times when given certain QoS
% characteristics and network load. A queueing system \cite{21} can be used to
% model this problem. An M/G/1 queueing model with vacations \cite{20} is, we
% claim, suitable for such a problem.

%Another challenge is to deploy and simulate a query-based WSN. A full
% deployment of transaction handling distributively in WSNs would include transaction processing and optimization, routing, aggregation, etc. \cite{2}. We will start with a simple query-based scheme in our deployments. However, we will design it so it would be possible to incrementally develop it to include more transactional features.

*/