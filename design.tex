\section{System design}\label{sec:design}

\subsection{Message Queue Manager}
The message queue manager provides a highly scalable and highly available message queue(s). Each workflow instance
  has atleast one instance of a message that contains the status and instance
  information of a workflow, and it triggers new jobs when it is consumed by a worker.
  MQM receives a message over the http protocol and removes a message from the
  queue when there exists a consumer that has an open channel with it. The MQM maintains
  two sets of queues - an active queue for all the available messages, and a
  queue for unacknowledged messages. Any message that is consumed by a consumer
  is not removed from the MQM until it has been acknowledged. When a message is
  requested by a worker for processing, it is transferred from the active queue
  to unacknowledged queue. If it is acknowledged, the message is served, else it
  is requeued back into the queue. MQM provides high scalability by providing
  mechanisms for topic based clustering of messages. We can implement multiple
  queues based on some partitioning mechanism, that can each serve messages for a cluster of workers.
\subsection{Worker}
%/* Worker definition*/
A worker is where the task execution occurs. A worker polls to obtain a task
from the Message Queue Manager. Once a new message is received, it obtains the
workflow run id and status from the message. The worker requests for the
workflow metadata from the repository, after which it executes the next task. After the task processing
is complete, it writes back the workflow run state to the repository, data
changes are saved and creates a new message that is sent to Message Queue
Manager. 
\subsection{Workflow}
As we defined earlier as well, a workflow is a set of tasks. Each task operates
upon a single key-value pair. Before the workflow executes the business logic
associated with itself, it creates a key group of all the keys (entities) that
are accessed by the workflow. We consider this step as a blackbox and use the
same protocol as presented in [G-Store]. If at any step, a task fails, then the
workflow reverses the effect of the partially executed workflow by running the
negate tasks, with the last executed task first. The workflow completes by
either successfully completing all the tasks, including key group creation and
deletion, or by successfully negating the effects of any partial run. 
