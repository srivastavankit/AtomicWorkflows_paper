\section{Related Work} \label{sec:relwork}

Google's datastore allows creation of transactions across multiple
entities (keys) as in []. Although, the only restriction with their
approach is static grouping of entities that restrict re-usability of an entity for another
transaction. Amazon S3 guarantees atomicity at the granularity of single
key-value pair. The transactional semantics for creating a transaction across
single/multiple buckets(entities) is left to the application developer. It can,
although, be easily achieved by using a simple two-layered close nested
distributed transaction model as described in [http://www.dbis.prakinf.tu-ilmenau.de/publications/files/DBIS:GroSat11.pdf].
---Add scalaris way of doing a transaction while maintaining ACID---.

The notion of transaction as a workflow has been extensively researched and a
lot of executional models have already been proposed in
as in [Colored Petri Nets, Divy's Paper, TRWF Paper]. But, this is the first
attempt that researches into the issue of a transaction in the context of a
distributed key value storage setting where the semantics of ACID are not
dropped in order to abide by the CAP theorem. This is the first attempt, as far as we know,
that conceptualizes a new model of executing transactions across multiple
entities (keys) on an underlying distributed key value store, where the most
granular set of operation, currently, is a single key update. Thereby, still maintaining
the ACID properties in a distributed setting, without incurring the overhead involved
with carrying out distributed transactions[some paper].

G-Store[link] extends the functionalities of a key-value datastore by allowing
creation of dynamic key (entity) groups that can be leveraged for transactional
multi-keys (entities) access, while maintaining a low overhead in group
creation stage. 
