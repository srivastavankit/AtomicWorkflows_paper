\section{Introduction}\label{sec:Intro}
Distributed key-value systems have been widely accepted as the solution to
provide high scalability, fault tolerance and availability to data intesive web
applications. These new age database management systems - Google's Bigtable,
Amazon's Dynamo, Facebook's Cassandra, et al - aim to provide support for a wide
variety of applications - search indexing, social network, and even an
e-commerce portal by providing easy-to-access, highly scalable and simple key
value storage. Although, this comes with the cost of increased complexity at the
application developement layer for implementing transactional semantics. Since all of these key-value
stores, with a few exceptions (like Google's Datastore), provide atomic
transactions at a single key (entity) level, they tend to diverge away from the notions of ACID.
Google's datastore tries to solve this problem by providing multi-key
transactions at the cost of static key grouping[]. G-Store presents dynamic,
on-demand key group creation that can be used as an abstraction for transactional access[].
Although, distributed protocols - distributed 2-phase commit and locking -
exhibit very high overhead [modify] that further restrict the usability of
distributed transactions.

As per CAP theorem[1], it is not possible to provide
availability and strong consistency guarantees in presence of network
failures[2]. Coping up with the CAP theorem, most of the distributed key value
stores have dropped some or most of ACID semantics, that has long been accepted
as the notion for providing and validating transactional semantics in a data
management system. Google's datastore provides strong consistency,
isolation and durability at the cost of limiting the scope of atomicity.
Atomic transactions are limited to either single key transactions or in
case of multi-keys where the keys belong to a key group that is static
in nature. Dropping some or most of these properties loosens the correctness
criteria of these transactions[]. New age applications of these distributed
storage systems - like banking, scientific simulations, etc - require
atleast same or more stricter guarantees as provided by ACID properties, going
forward.  

Pat Helland in his foresightful paper - Life beyond Distributed Transactions: an
Apostate�s Opinion []- presented the idea of modelling a distributed transaction
as a workflow. In his paper, he asserted that most of the real life transactions
are in fact workflows, where the management of uncertainity must be handeled along with business logic. In this paper, we
attempt at modelling distributed transactions as workflows to handle
the semantics of uncertainity. A workflow is a set of tasks, that according to
our model is atomic. Either the whole workflow is executed, or nothing is
executed as in case of failures.  This is carried out by nullifying the
effects of partially run workflows by creating tasks that negate the effects of
already executed tasks. We believe that not all real world tasks are
-reversible-, and require external or new tasks to negate its effects. For
example, a task of issuing a bank check is negated by creating a task to cancel
the validity of the check. Once issued and dispatched, the check can not be
retrieved back, although its validity can be easily nullified.

We have three major contributions to this paper -
\begin{itemize}
\item{We formalize the notion of a transaction as a workflow in a distributed
setitng.}
\item{We present a system that can be deployed for running such
distributed workflows that aim at maintaining ACID semantics, while emulating
the behavior of atomic, distributed transactions.}
\item{We implement a messaging based system that is designed to handle the
semantics of uncertainity. }
\end{itemize}

